\documentclass{sig-alternate-ipsn13}

\begin{document}

\title{Mailbox}
%
% You need the command \numberofauthors to handle the 'placement
% and alignment' of the authors beneath the title.
%
% For aesthetic reasons, we recommend 'three authors at a time'
% i.e. three 'name/affiliation blocks' be placed beneath the title.
%
% NOTE: You are NOT restricted in how many 'rows' of
% "name/affiliations" may appear. We just ask that you restrict
% the number of 'columns' to three.
%
% Because of the available 'opening page real-estate'
% we ask you to refrain from putting more than six authors
% (two rows with three columns) beneath the article title.
% More than six makes the first-page appear very cluttered indeed.
%
% Use the \alignauthor commands to handle the names
% and affiliations for an 'aesthetic maximum' of six authors.
% Add names, affiliations, addresses for
% the seventh etc. author(s) as the argument for the
% \additionalauthors command.
% These 'additional authors' will be output/set for you
% without further effort on your part as the last section in
% the body of your article BEFORE References or any Appendices.

\numberofauthors{1} %  in this sample file, there are a *total*
% of EIGHT authors. SIX appear on the 'first-page' (for formatting
% reasons) and the remaining two appear in the \additionalauthors section.
%
\author{
% You can go ahead and credit any number of authors here,
% e.g. one 'row of three' or two rows (consisting of one row of three
% and a second row of one, two or three).
%
% The command \alignauthor (no curly braces needed) should
% precede each author name, affiliation/snail-mail address and
% e-mail address. Additionally, tag each line of
% affiliation/address with \affaddr, and tag the
% e-mail address with \email.
%
% 1st. author
\alignauthor
Chun-Yen Hsu\\
       % \titlenote{Dr.~Trovato insisted his name be first.}\\
       \affaddr{Carnegie Mellon University}\\
       \affaddr{94035 Mountain View}\\
       \affaddr{California, USA}\\
       \email{chunyenh@andrew.cmu.edu}
% Ben Trovato\titlenote{Dr.~Trovato insisted his name be first.}\\
%        \affaddr{Institute for Clarity in Documentation}\\
%        \affaddr{1932 Wallamaloo Lane}\\
%        \affaddr{Wallamaloo, New Zealand}\\
%        \email{trovato@corporation.com}
% % 2nd. author
% \alignauthor
% G.K.M. Tobin\titlenote{The secretary disavows
% any knowledge of this author's actions.}\\
%        \affaddr{Institute for Clarity in Documentation}\\
%        \affaddr{P.O. Box 1212}\\
%        \affaddr{Dublin, Ohio 43017-6221}\\
%        \email{webmaster@marysville-ohio.com}
% % 3rd. author
% \alignauthor Lars Th{\o}rv{\"a}ld\titlenote{This author is the
% one who did all the really hard work.}\\
%        \affaddr{The Th{\o}rv{\"a}ld Group}\\
%        \affaddr{1 Th{\o}rv{\"a}ld Circle}\\
%        \affaddr{Hekla, Iceland}\\
%        \email{larst@affiliation.org}
% \and  % use '\and' if you need 'another row' of author names
% % 4th. author
% \alignauthor Lawrence P. Leipuner\\
%        \affaddr{Brookhaven Laboratories}\\
%        \affaddr{Brookhaven National Lab}\\
%        \affaddr{P.O. Box 5000}\\
%        \email{lleipuner@researchlabs.org}
% % 5th. author
% \alignauthor Sean Fogarty\\
%        \affaddr{NASA Ames Research Center}\\
%        \affaddr{Moffett Field}\\
%        \affaddr{California 94035}\\
%        \email{fogartys@amesres.org}
% % 6th. author
% \alignauthor Charles Palmer\\
%        \affaddr{Palmer Research Laboratories}\\
%        \affaddr{8600 Datapoint Drive}\\
%        \affaddr{San Antonio, Texas 78229}\\
       % \email{cpalmer@prl.com}
}
% There's nothing stopping you putting the seventh, eighth, etc.
% author on the opening page (as the 'third row') but we ask,
% for aesthetic reasons that you place these 'additional authors'
% in the \additional authors block, viz.
\additionalauthors{Additional authors: John Smith (The Th{\o}rv{\"a}ld Group,
email: {\texttt{jsmith@affiliation.org}}) and Julius P.~Kumquat
(The Kumquat Consortium, email: {\texttt{jpkumquat@consortium.net}}).}
\date{30 July 1999}
% Just remember to make sure that the TOTAL number of authors
% is the number that will appear on the first page PLUS the
% number that will appear in the \additionalauthors section.

\maketitle
% \begin{abstract}


% \end{abstract}

\section{Introduction}

Introduction

\section{Implementation} 
We developed a Smart Mailbox prototype by using infrared ray transmitter, camera, flash light, Arduino UNO board, Wifi model, general mailbox and several Arduino API. We put infrared ray transmitter behind the entry of the mailbox. At the beginning, infrared ray will launch and receive the reflection to know the depth of the mailbox. Whenever a new mail is put into the mailbox, the new mail will block the infrared ray which will make the detection depth significant decreases. In other words, the detection depth will be from the distance between the top and the bottom of the mailbox to the distance between the top of the mailbox and the new mail. This infrared sensing function can let us start to know the event of new mail coming in and launch the latter function in order to notify the user.

After the mail come in, we will send a flag from Arduino UNO board to the camera and the flash light to notify that the flash light can flash and the camera can capture a photo for new mail. After taking the photo, we want to send an email notification to the apartment host. We connect wifi module with Arduino board and use Arduino Email client API to send the email with the newest mail photo to the host's email address.


% Smart Mailbox is implemented by multiple

% \begin{figure}
% \centering
% \epsfig{file=fly.eps}
% \caption{A sample black and white graphic (.eps format).}
% \label{fig:example}
% \end{figure}

\section{Related Work}
Conclusion goes here.

\section{Evaluation-for Torque}
% \subsection{User Study 1: Creating Tutorials}
We performed user study and selected two type of wrench to make comparison with our connected torque management system. One is the general wrench and another is digital torque wrench on the market. The experience were classified into two tasks to finish, medium bicycle bolts and large size car bolts. The general wrench does not connect to the database so the user can not know the this information from the wrench. However, we slightly modify the process to provide this information in order for fair comparison. In addition, the general wrench and the digital torque wrench from the market don't provide the assembly process by itself. we also provide these information at the beginning. 

After user finished these two tasks, we will inquire users' feedback by using questionnaire combined with likert scale(from Strongly Like to Strongly Dislike). The data we want to conduct is whether torque management system and automatically procedures provide a better user experence or not. The questionnaire will also be analyzed by using analysis of variance (ANOVA) method. If the result shows that there is a significant difference for our new product, it means that our connected torque management system is significantly easier and better user experience. 


%ACKNOWLEDGMENTS are optional
% \section*{Acknowledgments}
% Acknowledgement goes here.

%
% The following two commands are all you need in the
% initial runs of your .tex file to
% produce the bibliography for the citations in your paper.
% \bibliographystyle{abbrv}
% \bibliography{sigproc}  % sigproc.bib is the name of the Bibliography in this case
% You must have a proper ".bib" file
%  and remember to run:
% latex bibtex latex latex
% to resolve all references
%
% ACM needs 'a single self-contained file'!
%
%APPENDICES are optional
%\balancecolumns
% \appendix
%Appendix A

% Appendix goes here.

% That's all folks!
\end{document}
